
\documentclass[letterpaper,10pt]{article}

% Language setting
% Replace `english' with e.g. `spanish' to change the document language
\usepackage[utf8]{inputenc}
\usepackage[T1]{fontenc}
\usepackage[english]{babel}

% Set page size and margins
% Replace `letterpaper' with `a4paper' for UK/EU standard size
\usepackage[letterpaper,top=2cm,bottom=2cm,left=3cm,right=3cm,marginparwidth=1.75cm]{geometry}

% Useful packages
\usepackage{amsmath}
\usepackage{amssymb}
\usepackage{stmaryrd}
%\usepackage{graphicx}

\usepackage{biblatex} %Imports biblatex package
\usepackage[colorlinks=true, allcolors=blue]{hyperref}

\addbibresource{bibliography.bib} %Import the bibliography file


\title{Exos sympas}
\author{Armand Perrin}

\begin{document}

\maketitle%---------------------------------------------------------------
\newcommand{\m}[1]{\ensuremath{\mathbb{#1}}}
\newcommand{\tx}[1]{\ensuremath{\mathrm{#1}}}

\newcommand{\mn}[1]{\ensuremath{\mathcal{M}_n(\mathbb{#1})}}
\newcommand{\Z}{\ensuremath{\mathbb{Z}}}
\newcommand{\car}[1]{\ensuremath{\chi_{_{#1}}}}
\newcommand{\n}[1]{\ensuremath{\|#1\|}}
\newcommand{\gln}[1]{\ensuremath{GL_n(\mathbb{#1})}}
\setlength\parindent{0pt}
\newcommand{\exercice}[3]{
  \fbox{%
\begin{minipage}{1.0\textwidth}
  \paragraph{Exercice #1:} #2

\end{minipage}
}
  \paragraph{Solution :} #3 $ \\ \\ \\$
}%-------------------------------------------------------------------------------------
\exercice{1}{
Soit $(G,.)$ un groupe fini dont tous les élément sont d'ordre 2, montrer que son
cardinal est une puissance de 2.}
{On propose 2 solution, la première très astucieuse et jolie et la deuxième plus accessible et généralisable.
\paragraph{Solution 1:}
Cette solution consiste à remarquer que $(G,\cdot,\wedge)$ est un $\Z/2\Z$-espace vectoriel (ça a un sens car $\Z/2\Z$ est un corps).\\
Où la loi externe $\wedge$ est définie par 
$ \wedge: \left\{ \begin{array}{rcl} \Z/2\Z\times G& \longrightarrow & G \\
   (n,x) & \longmapsto & x^n \\ \end{array} \right. $
 En effet : \\ - $ (G,.)$ est un groupe abélien : 
Soient $x,y \in G\quad  1 = (xy)^2 = xyxy $ et en composant a gauche par 
$x$ et a droite par $y$ on a bien $xy = yx$.\\
- $\forall x\in G \quad x^1 = x$\\
- $\forall x,y \in G, n\in \Z/2\Z \quad (xy)^n = x^ny^n$ car $G$ est abélien.\\
- $\forall n,m \in \Z/2\Z, x\in G \quad x^{n+m} = x^nx^m$\\
- $\forall n,m \in \Z/2\Z, x\in G \quad (x^n)^m = x^{nm}$\\
cet espace vectoriel est de dimension finie  car il est fini ( tout famille de taille supérieure à $card(G)$ est liée). Notons $k$ sa dimension, la fixation d'une base de $G$ induit un isomorphisme entre $G$ et $(\Z/2\Z)^k$ ils sont donc de même cardinal :  $2^k$.
\setlength\parindent{0pt}
\paragraph{Solution 2:}
On démontre par récurrence sur $k$ la propriété : ``pour tout groupe $G$ dont tous les éléments sont d'ordre 2 : $card(G) \leq 2^k \implies \exists p \in \mathbb{N}$ tel que $card(G) = 2^p$ ''. Avoir cette propriété pour tout $k$ permet clairement de conclure. \\
Initialisation :  si $k = 0$ c'est bon.\\
Hérédité : supposons la propriété réalisée pour $k \geq 1$, soit $G$
un groupe dont tous les éléments sont d'ordre 2 tel 
que $card(G) \leq 2^{k+1}$ , si $card(G) \leq 2^k$ c'est bon par 
l'hypothèse, sinon soit $H$ un sous groupe strict de $G$ de cardinal 
maximal (existe car ils sont en nombre fini) et $a \in G\setminus H$ 
alors $G = H \cup aH$ et $H\cap aH=\emptyset$, en effet $H\cup aH$ est 
un sous groupe de $G$ :
\par\leftskip 10pt
-$ 1 \in H\cup aH$\\
-Soient $x,y \in H\cup aH$
\par\leftskip 20pt
si $x,y \in H$ alors $xy^{-1} \in H \subset H\cup aH$\\si $x,y\in aH\quad \exists h,h'\in H \quad xy^{-1} = ah(ah')^{-1} = hh'^{-1} \in H$ car $G$ est commutatif (voir Solution 1)\\ si $x\in H,y\in aH$ (ou le contraire, par symétrie) $\exists h,h'\in H \quad xy^{-1} = h(ah')^{-1} = ahh'^{-1} \in aH$ car $a^{-1} =a$.\par
\leftskip 0pt
Alors $H\cup aH = G$ par maximalité de $H$. Soit $x\in H \cap aH \quad \exists h,h' \in H \quad h' = ah$ absurde car alors $a= h'h^{-1} \in H$. Ainsi $H\cap aH=\emptyset$. En vertue de la bijection $x \mapsto ax,\quad card(H) =card(aH)$. On déduit de ces 3 points que $card(G) = 2card(H)$, donc $card(H) \leq 2^k$ et par l'hypothèse de récurrence $\exists p \in \mathbb{N} \quad card(H) = 2^p$ puis $card(G) = 2^{p+1}$.
}
\exercice{2 (d'après \cite{igor})}%---------------------------------------------------------------
{
  Soit $\mathbb{K}$ un corps et $A_1, \ldots,A_p$ un ensemble de $p$ matrices de $GL_n(\mathbb{K})$ stable
par produit, montrer que \[\ \tx{Tr}\Biggl(\sum_{i=1}^p A_i\Biggr) \equiv 0\left[p\right] \]
}
{
  Notons $G$ cet ensemble et $S = \sum_{i=1}^pA_i$, il faut remarquer que si $j \in \llbracket 1 , p \rrbracket$ 
\[\ \sum_{i=1}^pA_jA_i = \sum_{i=1}^pA_i = S \text{ car } M \mapsto A_jM \text{ est une permutation de } G \] \[\ \text{Alors, }
S^2 = \Biggl(\sum_{i=1}^pA_i\Biggr)^2 =  \sum_{i=1}^p\sum_{j=1}^pA_iA_j = p\sum_{i=1}^pA_i = pS
\] Si la caractéristique de $\m{K}$ divise $p$ (c'est à dire si $p.1_{\m{K}} = 0_{\m{K}}$) alors $S^2=0$, $S$ est donc nilpotente et sa trace est nulle.
On a bien $\tx{Tr}(S) \equiv 0 \left[p\right] $.   \[\ \text{Sinon : }\Bigl(\frac{S}{p}\Bigr)^2 = \frac{pS}{p^2} = \frac{S}{p} \] $\frac{S}{p}$ est donc un projecteur,  sa trace est donc égale à son rang, c'est donc un entier et par linéarité, la trace de S est divisible par $p$. 

}

\exercice{3}{
Soient $a,b\in \mathbb{R},\;\; a<b$ et $f:\left[a,b\right] \to \mathbb{R}$ continue 
sur $\left[a,b\right]$, dérivable sur $\left]a,b\right[$ nulle en $a$ et $b$.
 Soit $\lambda \in \mathbb{R}$.
Démontrer que $f' + \lambda f$ s'annule sur $\left]a,b\right[$. 

}{

Les hypothèses ressemblent à celles du théorème de Rolle, on va chercher à l' appliquer, mais à quelle fonction ? \\
Analyse : On va chercher par exemple $g$ dérivable vérifiant \\
- $g(a) = g(b)$\\
- $g'(c) = 0 \implies f'(c) + \lambda
f(c) = 0$\\
le deuxième point serait vérifié si par exemple $g'= (f'+\lambda f)h$ avec $h$ une fonction strictement positive. Choisissons $h$ de manière à pouvoir primitiver $g'$ on remarque que $g'$ est la dérivée du produit $fh$ si $h' =\lambda h$ ce qui fonctionne avec $h : x\mapsto e^{\lambda x}$.\\
Synthèse : On applique le théorème de Rolle à $g:x\mapsto f(x)e^{\lambda x}, g $ est continue sur $\left]a,b\right[$, dérivable sur $\left[a,b\right]$, nulle en $a$ et $b$ donc $ \exists c \in \left]a,b\right[ \quad g'(c) = 0 $ or $g'(x) = (f'(x) + \lambda f(x))e^{\lambda x}$ donc $f'(c) + \lambda f(c) = 0$. 
}
\exercice{4}{
    Retrouver le binome de Newton à l'aide de la formule de Leibniz.
}{
    Si $a\in \mathbb{C}$ on cherche une fonction $f$ de classe $\mathcal{C}^{\infty}$ pour laquelle on ait une relation simple entre $f^{(k)}$ et $a^k$ pour tout $k$. Dans la lignée de l'exercice précédant on va considerer les fonctions exponentielles, posons : $f_a : x \mapsto e^{ax}$ alors $f_a^{(k)} = a^kf_a$.
    Soient $a,b \in \mathbb{C},n\in \mathbb{N}$ on a avec Leibniz : \[\
    (a+b)^n = (a+b)^ne^{(a+b)\times 0} =f_{a+b}^{(n)}(0) = (f_af_b)^{(n)}(0)\] \[\ = \sum_{k=0}^n\binom{n}{k}f_a^{(k)}(0)f_b^{(n-k)}(0) = \sum_{k=0}^n \binom{n}{k}a^kb^{n-k}
    \]
}
\exercice{5 (Bézout dans \mn{Z})}{
Soient $A,B \in \mn{Z}$ telles que $ det(A) \wedge det(B) = 1$, montrer qu'il
existe $U,V \in \mn{Z}$ telles que : \[\ AU+BV = I_n\]
}
{
C'est direct quand on pense à la formule $MCom(M)^{T}= det(M)I_n$,
car une relation de Bézout dans $\mathbb{Z}$ permet d'écrire : \[\ 
\exists u,v \in \mathbb{Z} \quad ACom(A)^{T}u + BCom(B)^Tv = (det(A)u+det(B)v)I_n = I_n
\]
}
\exercice{6}{
Soit $M \in \mn{C}$, démontrer que $M$ est diagonalisable si et seulement si
sa classe de simlitude est fermée.
}{
  $\boxed{\Rightarrow }$ Supposons $M$ diagonalisable,\\ on note $C = \{PMP^{-1} , P \in GL_n(\mathbb{C}) \}$ la classe de similitude de $M$, démontrons qu'elle est fermée. Soit $(M_n)_n \in C^{\mathbb{N}}$ convergente de limite $L \in \mn{C}$.
Notons $\Pi_M$ le polynome minimal de $M$ et $\chi_{_M}$ son polynome caractéristique. L'applcation $M \mapsto det(XI_n -M) = \chi_{_M}$ étant continue car polynomiale, on a $\chi_{_{M_n}} \rightarrow \chi_{_L}$ or $\forall n \in \mathbb{N} \quad \chi_{_{M_n}} = \chi_{_M}$ donc $\chi_{_M} = \chi_{_L}$. De plus, on a classiquement pour $P \in GL_n(\mathbb{C}) \quad \Pi_M(PMP^{-1}) =  P\Pi_M(M)P^{-1} = 0$. Donc $\forall n\in \mathbb{N} \quad \Pi_M(M_n) = 0$ et par continuitée de $\Pi_M : \mn{C} \to \mn{C}$ comme polynome, $\Pi_M(L) = 0$ or $\Pi_M$ est scindé, à racines simples puisque $M$ est diagonalisable, donc $L$ est aussi diagonalisable. $M$ et $L$ ont le même polynome caractéristique et sont diagonalisables donc semblables a une même matrice diagonale, elles sont donc semblables entre elles et $L \in C$.\\
$\boxed{\Leftarrow}$ Supposons que la classe de similitude $C$ de $M$ est fermée. On veut démontrer que $M$ est diagonalisable, ce qui équivaut à dire que $C$ contient une matrice diagonale. Il suffit donc de trouver une suite d'éléments de $C$ convergeant vers une matrice diagonale, puisque $C$ est fermée. $M$ est trigonalisable dans $\mathbb{C}$ donc on peut trouver $T \in C$ triangulaire supérieure. Considérons pour $k \in \mathbb{N}^*$ la matrice digonale inversible : $P_k = \left( \begin{matrix}k & 0 & \ldots & 0 \\ 0 & k^2  & \ldots & 0 \\ \vdots & \vdots & \ddots & \vdots \\ 0 & 0& \ldots & k^n \end{matrix}\right)$ d'inverse $\left( \begin{matrix}\frac{1}{k} & 0 & \ldots & 0 \\ 0 & \frac{1}{k^2}  & \ldots & 0 \\ \vdots & \vdots & \ddots & \vdots \\ 0 & 0& \ldots & \frac{1}{k^n} \end{matrix}\right)$. La suite $(U_k)_k= (P_kTP_k^{-1})_k$ de $C$ converge vers $diag(t_{1,1},\ldots,t_{n,n})$ où $T =(t_{i,j})_{1\leq i,j \leq n}$. En effet le coefficient d'indice $(i,j)$ de $U_k$ vaut $t_{i,j}k^{i-j}$. Si $i>j$ alors $t_{i,j} =0$, si $i<j\quad k^{i-j} \to 0$ et si $i=j\quad t_{i,j}k^{i-j} \to t_{i,i}.\;\;_{_\square}$
\paragraph{Remarque : } On peut démontrer avec la même méthode que $M$ est nilpotente si et seulement si $0 \in \overline{C}$ (l'adhérence de $C$).
}
\exercice{7 (Théorème de Maschke)}{
Soit $E$ un espace vectoriel de dimension finie et G un sous groupe fini de 
$GL(E)$. Démontrer que tout sous espace vectoriel de $E$ stable par tous les éléments de $G$ admet un 
supplémentaire également stable par tout les élément de $G$.
}{
  Disons d'un sous-espace qu'il est stable par $G$ si il est stable par tous les éléments de $G$.
Soit $F$ un sous-espace vectoriel de $E$ stable par $G$. On va travailler avec des projecteurs, il est bien de se rendre compte que à un couple de sous espaces supplémentaire on peut toujours faire correspondre un projecteur et vice versa car le noyeau et l'image d'un projecteur sont supplémentaires. Soit donc $p$ un projecteur sur $F$. L'idée pricipale pour construire un supplémentaire stable par $G$ est de chercher à construire un projecteur à partir de $p$ et $G$ ayant aussi $F$ pour image et dont le noyeau est stable par $G$.
On a déja remarqué dans l'exercice 2 que la somme des éléments d'un sous groupe fini de $GL_n(\mathbb{K})$ divisée par son cardinal est un projecteur, on peut donc s'en inspirer et poser $s = \frac{1}{n} \sum_{g \in G}^n gpg^{-1}$, où $n= card(G)$,
de cette manière : \[\ s^2 = \frac{1}{n^2} \sum_{g \in G}^n\sum_{h \in G}^n hph^{-1}gpg^{-1} = \frac{1}{n}\sum_{g \in G}^ngpg^{-1} =s \] car $p_{|F} =id_F$ et $F =Im(p)$ est stable par $G$. $s$ est le projecteur recherché, on a $Im(s) = F$ en effet :\\ $\boxed{\subset}$ Cette incluson est claire pusique $F$ est stable par $G$.\\ $\boxed{\supset}$ si $x\in F$  alors $ x = p(x) = \frac{1}{n}\sum_{g \in G}^n gpg^{-1}(x) \in Im(s)$ car $p_{|F} =id_F$.\\
On a aussi $Ker(s)$ stable par $G$ : soit $x\in Ker(s)$, et $h \in G$ \\
\[\ s(h(x)) = \frac{1}{n} \sum_{g\in G}^n gpg^{-1}(h(x)) = \frac{1}{n} \sum_{g\in G}^n hh^{-1}gp(h^{-1}g)^{-1}(x) = hs(x) = 0\] car $g \mapsto h^{-1}g$ est une permutation de $G$. Donc $Ker(s)$ est stable par $G$. Finalement comme $ Ker(s) \oplus F = Ker(s) \oplus Im(s) = E,\;\; Ker(s)$ est le supplémentaire recherché.\\ 
\\ $Remarque :$ Cette preuve est vraie pour tout corps $\mathbb{K}$ tel que $n.1_{\mathbb{K}}\neq 0$ (pour pouvoir diviser par $n$), c'est à dire dont la caractéristique ne divise pas $n$.\\
\paragraph{Autre solution si $\mathbb{K} =\mathbb{R}$ :} On identifie $E$ et $\mathbb{R}^p$ que l'on munit du produit scalaire canonique $<.,.>$, et on pose le produit scalaire $(.|.)$ sur $E$ définit par : \[\ \forall x,y \in E \quad (x|y) = \sum_{g\in G}^n <g(x),g(y)> 
\] On travaille dans l'espace euclidien $(E,(.|.)) $, encore par la bijection $g\mapsto hg$ pour $h \in G$, on constate que tout les éléments de $G$ conservent le produit scalaire, donc sont autoadjoints et donc $F^{\perp}$ est un supplémentaire de $F$ stable par $G$.
}
\exercice{8}{ Démontrer que $GL_n(\mathbb{C})$ est connexe par
 arcs.
}{ Soit $A \in GL_n(\mathbb{C})$
On va exhiber une application continue $\phi : \left[ 0,1 \right] \to GL_n(\mathbb{C})$ telle que $\phi(0) = I_n$ et $\phi(1) = A$ 

Regardons pour commencer le chemin en ligne droite : $\psi_A(t) = tA + (1-t)I_n$ il n'a pas de raison de rester dans $GL_n(\mathbb{C})$ mais pour \[\ t \in \left] 0,1\right] \quad \psi_A(t) \in GL_n(\mathbb{C})\quad \Leftrightarrow \quad A + \frac{1-t}{t}I_n \in GL_n(\mathbb{C}) \Leftrightarrow \frac{1-t}{t} \notin Sp(A) \] Il suffit donc que $A$ n'ait aucune valeur propre réelle pour que ce chemin $\psi_A$ soit à valeurs dans $GL_n(\mathbb{C})$. Que faire si ce n'est pas le cas ? Trouver un chemin continu de $A$ à une matrice dont les valeurs propres ne sont pas réelles. Posons \[\  
r:\left\{ \begin{matrix}
 \left[0,1\right] &
  \longrightarrow & GL_n(\mathbb{C}) \\  t & \longmapsto &  e^{2i\pi t}A \end{matrix} \right. \] 
Si $Sp(A) = \{\lambda_1,\ldots,\lambda_n \}$ alors $Sp(r(t)) = \{e^{2i\pi t}\lambda_1,\ldots,e^{2i\pi t}\lambda_n\}$.\\ Puisque   l'ensemble $\{ t\in \left[ 0,1\right] | \exists k \in \llbracket 1,n \rrbracket \quad e^{2i\pi t}\lambda_k \in \mathbb{R}\}$ 
est fini on peut trouver  $t_0\in \left[0,1\right]$ 
tel que $Sp(r(t_0)) \subset \mathbb{C}\setminus \mathbb{R}$. Notons $B = r(t_0) $. La restriction de $r$ à $ \left[0,t_0\right]$ est donc un chemin continu à valeurs dans $GL_n(\mathbb{C})$ joignant $A$ et $B$, de même $\psi_B$ continu à valeurs dans $GL_n(\mathbb{C})$ joint $B$ et $I_n$, on construit donc facilement $\phi $ comme il faut.}
%------------------------------------------------------------------------------------------------------------------------------------------------

\exercice{9 (ENS Lyon 2014)}{

Determiner une CNS sur $A$ et $B$ dans $\mn{C}$ pour que \[\
\forall k \in \mathbb{N} \quad tr(A^k) = tr(B^k) \quad (*) \]
}
{
Pour commencer simplifions le problème et cherchons les matrices $A$ qui vérifient \[\
\forall k \in \mathbb{N^*}\quad tr(A^k) = 0
\quad (**) \] C'est classique, on commence par observer que toutes les matrices 
nilpotentes fonctionnent, puis on montre que ce sont les seules :
 démontrons que si $A \in \mn{C}) $ verifie $(**)$ alors son
  spectre est réduit à \{0\}. Notons $ \lambda_1,..., \lambda_r$ les valeurs propres non nulles de $A$ où les $\lambda_i$ sont distinct 2 à 2, $n_i \geq 1$ la multiplicité de $\lambda_i$ et $n_0 \geq 0$ la multiplicité de 0. On raisonne par l'absurde en supposant $r \geq 1$,
$(**)$ pour $k\in \llbracket 0, r \rrbracket $ donne : 
\begin{alignat*}{6}
n_0 + &n_1\lambda_1^0 &+\ldots &+n_r\lambda_r^0 & = n\\
&n_1\lambda_1 &+\ldots&+n_r\lambda_r  &= 0\\
&n_1\lambda_1^2 &+\ldots&+n_r\lambda_r^2 &= 0 \\
&\vdots &\vdots  & & \vdots \\
&n_1\lambda_1^{r} &+\ldots&+n_r\lambda_r^{r} &= 0
\end{alignat*}
Donc 
\[\ \begin{pmatrix}
1 & 1 &\ldots & 1 \\
0^1 & \lambda_1^1 & \ldots & \lambda_r^1 \\
\vdots & \vdots & \vdots & \vdots \\
0^r & \lambda_1^r & \ldots & \lambda_r^r \\
\end{pmatrix}\begin{pmatrix}n_0-n \\ n_1\\ \vdots \\ n_r
\end{pmatrix}
=\begin{pmatrix}0 \\ 0\\ \vdots \\ 0
\end{pmatrix}\]
Il s'agit d'un système type Vandermonde  inversible,
on en déduit $\forall i \in\llbracket 1, r\rrbracket \;\; n_i = 0 $ ce qui est absurde, donc $r = 0$ et $Sp(A) =$ \{0\}

Pour le cas général ont peut remarquer que si $A$ et $B$ ont le même polynome
 caractéristique alors elles vérifient $(*)$ de plus la réciproque est vraie 
 si $B=0$ d'après ce qui précède. On va montrer qu'elle est toujours vraie. 
 Pour une matrice $M$ notons $\chi_{_M}$ son polynome caractéristique et $\Pi{_M}$ 
 son polynome minimal. Fixons $A$ et $B$ dans $\mn{C})$ vérifiant $(*) $.
  D'abord, la linéarité de la trace permet d'écrire : \[\ \forall P \in 
  \mathbb{C}\left[X\right] \quad tr(P(A)) = tr(P(B)) \] 
Soit $k \in \mathbb{N}^*$ pour  $P = \chi_{_B}^k$ on a donc 
d'après Cayley-Hamilton : \[\ tr(\chi_{_B}^k(A)) = 0 \] La matrice
 $\chi_{_B}(A)$ est donc nilpotente, donc $\chi_{_B}^n$ annule $A$, donc
  $\Pi_{A} | \chi_{_B}^n$ et donc $Sp(A) \subset Sp(B)$. Par symétrie on 
  obtient ensuite l'inclusion réciproque, donc $Sp(A) = Sp(B)$.

Il ne reste plus qu'a montrer que les multiplicités des valeurs propres sont les
 mêmes pour avoir $\chi_{_A} =\chi_{_B}$. Notons $\{\lambda_1,\ldots,\lambda_r\}$ 
 le spectre commun et $a_i, b_i$ la multiplicité de $\lambda_i$ dans $\chi_{_A}$ 
 et $\chi_{_B}$ respectivement. $(*)$ donne alors :
\begin{alignat*}{8}
 &a_1\lambda_1^0 &+ \ldots  +& a_r\lambda_r^0 &=&\; b_1\lambda_1^0 &+ \ldots  +& b_r\lambda_r^0 \\
&\vdots & & & \vdots & & \vdots &\\
 &a_1\lambda_1^{r-1} &+ \ldots  +& a_r\lambda_r^{r-1} &=&\; b_1\lambda_1^{r-1} &+ \ldots  +& b_r\lambda_r^{r-1} \\
\end{alignat*} Ce qui s'écrit :
\[\ \begin{pmatrix}
1  &\ldots & 1 \\
\lambda_1^1 & \ldots & \lambda_r^1 \\
\vdots & \vdots & \vdots \\
\lambda_1^{r-1} & \ldots & \lambda_r^{r-1} \\
\end{pmatrix}\begin{pmatrix}a_1-b_1 \\ a_2-b_2\\ \vdots \\ a_r-b_r
\end{pmatrix}
=\begin{pmatrix}0 \\ 0\\ \vdots \\ 0
\end{pmatrix}\] On retombe sur une matrice de Vandermonde inversible et on conclut que \[\ \forall i \in \llbracket 1,r \rrbracket \quad a_i= b_i \] ce qui conclut.
}
%------------------------------------------------------------------------------------------------------------------------------------------------

\exercice{10 (Cassini)}{Soit $g\in \mathcal{C}^0(\left[ a, b\right],\mathbb{R})$. On pose $f_0 = g$ et \[\ \forall n \in \mathbb{N} \quad f_{n+1} = \int_af_n \] Étudier la convergence de la série de fonctions : \[\ \sum_{n\geq 0}f_n \] et calculer sa somme.

}{
%solution_____
Remarquons que $\forall n \in \mathbb{N} \quad f_n$ est de classe $ \mathcal{C}^n$ et $f_n^{(k)}  = f_{n-k}$ et si $n \neq 0$ $ f_n(a)=0$.
$g$ est bornée sur le segment $\left[ a,b\right]$ car elle y est continue notons $M$ une borne de $g$. Soit $x\in \left[ a,b\right]$ le théorème de Taylor-Lagrange appliqué à $f_n$ entre $a$ et $x$ s'écrit :
\[\ \Bigg|f_n(x)-\sum_{k=0}^{n-1}\frac{f_n^{(k)}(a)}{k!}(x-a)^k\Bigg|\leq \frac{M(x-a)^n}{n!} \]\[\ |f_n(x)|\leq \frac{M(b-a)^n}{n!}\] Par passage au $sup$
 on déduit que la série de terme général $\|f_n\|_{\infty} $ converge. Donc $\sum_{n\geq 0}f_n$ converge normalement donc uniformément sur $ \left[ a,b\right]$.\\
Calculons maintenant sa somme, que l'on note $s$ :\\
Soit $x \in \left[ a,b\right]$ : 
\[\ \int_a^xs(t)dt = \int_a^x\sum_{n=0}^{\infty}f_n(t)dt\]

La convergence uniforme nous autorise ici à échanger série et intégrale :

\begin{align*} \int_a^xs(t)dt &=  \sum_{n=0}^{\infty}\int_a^xf_n(t)dt \\ &= \sum_{n= 0}^{\infty}f_{n+1}(x) \\ &= s(x) -g(x) \\ \text{Notons } S(x) &= \int_a^xs(t)dt. \end{align*}

Alors $S' - S = g$, équation différentielle facile à résoudre, on obtient avec la condition $S(a) =0$ que  : \[\ S(x) = e^{x}\int_a^xg(t)e^{-t}dt \] et 
\[\ s(x) = g(x) + e^{x}\int_a^xg(t)e^{-t}dt  \]}
%------------------------------------------------------------------------------------------------------------------------------------------------

\exercice{11}{
Soit $A\subset \mathbb{C}$, exhiber un $\mathbb{K}$-espace vectoriel pour  $\mathbb{K} = \mathbb{R}$ ou $\mathbb{C}$ et $f\in L(E)$ tel que $Sp(f) = A$.
% Create a new 1st level heading
}{
Considérons $E = \mathcal{C}^{\infty}(\mathbb{R},\mathbb{C})$ et $D:f\mapsto f'$ l'endomorphisme de dérivation sur $E$, puis pour
 $\lambda \in \mathbb{C}\quad f_{\lambda}:x\mapsto e^{\lambda x}$.\\On pose $F = Vect_{\mathbb{C}}\Big((f_{\lambda})_{\lambda \in A}\Big)= 
 Vect_{\mathbb{R}}\Big((f_{\lambda})_{\lambda \in A}, (if_{\lambda})_{\lambda \in A} \Big)$ qui est dans tous les cas un sous espace vectoriel de $E$ vu comme $\mathbb{K}$-espace vectoriel, la suite de la preuve est similaire  que $\mathbb{K}$ soit $ \mathbb{R}$ ou $\mathbb{C}$ : \\

On constate que $F$ est stable par $D$, notons $d = D_{|F} \in L(F)$ : \[\ \forall \lambda \in A\quad d(f_{\lambda}) = \lambda f_{\lambda}\]
Et comme les $f_{\lambda}$ sont tous non nuls, $A \subset Sp(d)$.
Soit maintenant $\lambda \in Sp(d)$, \[\ \exists f \in F \quad f' = \lambda f \] On peut résoudre cette équation différentielle, 
et on obtient \[\ \exists K \in \mathbb{C}\quad \forall x \in \mathbb{R}\quad f(x) = Ke^{\lambda x} \] On a aussi $f \in F$, et la famille 
$(f_{\lambda})_{\lambda \in \mathbb{C}}$ est $\mathbb{K}$-libre, donc nécessairement $f = Kf_{\lambda}$ avec $\lambda \in A$. Ainsi $A = Sp(d)$.
}
\exercice{12 (Théorème de Burnside)}{
  On dit qu'un groupe $G$ est d'exposant fini si : \[\ \exists N \in \m{N}^* \quad \forall g \in G  \quad g^N = 1\] Le plus petit entier $N$   vérifiant cela est alors appelé l'exposant de $G$, c'est aussi le plus petit multiple commun des ordres des éléments de $G$.\\ \\ Démontrer qu'un sous groupe de $GL_n(\m{C})$ est d'exposant fini si et seulement si il est fini.
}{
  Il est clair que si $G$ est fini $\forall g \in G \quad g^{|G|} =1$ donc $G$ est d'exposant fini. 
  Réciproquement supposons $G$ d'exposant fini, on va montrer que $G$ est fini en construisant une injection de $G$ dans un ensemble fini. Soit $B = (B_1,\ldots,B_r)$ une famille libre de $G$ de cardinal maximal, on vérifie facilement que c'est une base de $Vect(G)$.\\
  Posons $ f: \left\{ \begin{array}{rcl} Vect(G) & \longrightarrow & \m{C}^r \\ A & \longmapsto & (Tr(AB_1),\ldots,Tr(AB_r)) \\ \end{array} \right. 
  $ $f$ est linéaire, montrons que $f_{|G}$ est injective: On vérifie facilement qu'il suffit pour ça de montrer que $Ker(f)\cap G =\{0\}$.  Soit $A \in Ker(f)\cap G$ et $k \in \m{N}$, on peut decomposer $A^k$ dans la base $B$, notons : \[\ A^k = \sum_{i=1}^r a_iB_i \] \[\ Tr(A^{k+1}) = \sum_{i=1}^r a_iTr(AB_i) = 0 \] Car $A \in Ker(f)$ donc \[\ \forall k \in \m{N}^*\quad Tr(A^k) = 0 \]
  Cela implique que $A$ est nilpotente (exercice classique, voir exo 9). Or le ploynome scindé à racines simples $X^N-1$ annule $A$ puisque $A \in G$, donc $A$ est diagonalisable et nilpotente, elle est donc nulle.
  $f_{|G}$ est donc injective.\\ Pour conclure montrons que $f(G)$ est fini. Déja $f(G) \subset E^r$ où\\ $E = \{Tr(A), A \in G\}$.
   Il suffit donc de montrer que $E$ est fini, et en effet si $A \in G$, les valeurs propres de $A$ sont des racines de $X^N-1$ qui sont en nombres fini, 
   la trace de $A$ est la somme de ses valeurs propres, donc peut également prendre un nombre fini de valeurs, ainsi $E$ est fini.
}
\exercice{13}{
  Déterminer les endomorphismes de groupe continus de $ (\m{R}^*_+,\times)$.
}{
  Soit $f$ un tel morphisme,
  posons $g = \mathrm{ln}\circ f\circ \mathrm{exp}$. On a \begin{align*}\forall x,y \in \m{R} \quad g(x+y) =&\; \mathrm{ln}(f(e^{x+y})) \\ =&\; 
    \mathrm{ln}(f(e^xe^y)) \\ =&\; \mathrm{ln}(f(e^x)f(e^y))\\ =&\; \mathrm{ln}(f(e^x)) + \mathrm{ln}(f(e^y)) \\ =& \; g(x) + g(y) \end{align*}
    L'application $g$ est continue comme composée, et additive, donc linéaire (c'est un exercice classique), donc il existe $a \in \m{R} $ tel que :
    \begin{align*}\forall x \in \m{R} \quad g(x) &= ax\\ \mathrm{ln}(f(e^x)) &= ax\\ f(e^x) &= e^{ax} \quad \text{par surjectivité de exp } 
      \forall y \in \m{R}  \quad \exists x \in \m{R}^{*}_+ \quad y = e^x \\ \text{donc } \forall y \in \m{R}^{*}_+\quad f(y) &= y^a \end{align*}

    Réciproquement, les fonctions de la forme $x\mapsto x^a$ avec $a \in \m{R}$ sont bien des endomorphismes continus de $(\m{R}^*_+,\times)$.
      \paragraph{Autre Solution :}
      Soit $f$ un tel morphisme, il vérifie : \[\ \forall x,y \in \m{R}^{*}_+ \quad f(xy) = f(x)f(y)\quad\quad  (1) \]
      En particulier $f(1)^2 = f(1)$ donc $f(1) \in \{0,1\}\cap \m{R}^{*}_+ $ donc $f(1) = 1$.
      On va chercher une équation différentielle vérifiée par $f$, pour montrer qu'elle est dérivable, intégrons (1) (bonne idée paradoxale) :
      puisque $f$ est continue et strictement positive : \[\ C := \int^2_1f(y)dy >0  \]
      puis, soit $ x\in \m{R}^{*}_+ $ :  \begin{align*}\int_1^2 f(xy)dy =&\; \int_1^2 f(x)f(y) dy \\
      \int_x^{2 x} f(u) \frac{du}{x} =&\; f(x)\int_1^2 f(y) dy\quad  \text{ avec le changement de variables } u = xy \\
       \text{d'où } f(x) =&\; \frac{1}{xC}\int_x^{2x}f(u) du
      \end{align*}
      $f$ est donc de classe $\mathcal{C}^1$ sur $\m{R}^*_+$.
      On peut donc maintenant dériver (1) par rapport à $x$ :
      \[\ \forall x,y \in \m{R}^*_+ \quad   yf'(xy) = f'(x)f(y) \]
      et en $x=1$ :
       \[\ \forall y \in \m{R}^*_+ \quad yf'(y) = f'(1)f(y) \] 
      En résolvant cette équation différentielle on obtient : \[\ \exists K \in \m{R} \quad \forall x \in \m{R}^*_+ \quad f(x) = Kx^{f'(1)}\]
      puis en évaluant en 1 il vient $K = 1$. Enfin on conclut car toutes les fonctions $x \mapsto x^a$ où $a \in \m{R}$ conviennent.

}
\exercice{14}{
Soit $ R >0 $ et $A$ une partie de la sphère de rayon R, que l'on note $ \\ \mathcal{S}(0,R) =\{z \in \m{C} \;\;|\;\; |z| = R \}$. 
Existe-t-il une série entiere $f:z\mapsto \sum_{n=0}^{\infty}a_nz^n$ de rayon de convergence $R$ tel que l'ensemble des points de $\mathcal{S}(0,R)$ en lesquels $f$ est bien définie (i.e la série converge) soit exactement $A$ ?
}{

On va démontrer que le résultat est faux en général par arguments de cardinalité : \\Soit $R>0$, pour une série entiere $f$ notons $C(f)$ l'ensemble des points de $\mathcal{S}(0,R)$ où $f$ converge. On raisonne par l'absurde en supposant que le résultat est vrai : pour tout  $A \subset \mathcal{S}(0,R)$ il existe une série entiere $f$ de rayon $R$ telle que $C(f) = A$.
On va commencer par démontrer le résultat suivant : 
\paragraph{Lemme :}
Si $(a_n)n$ est une suite de complexes, il existe une suite $(q_n)_n$ de $\m{Q}\left[i\right]$ telle que
 \[\ \forall z \in \m{C}\quad \Bigg(\sum_{n\geq 0}a_nz^n \;\text{ converge } \Leftrightarrow \quad \sum_{n\geq 0} q_nz^n \;\text{ converge}\Bigg)\]
\paragraph{Preuve :}
Soit $(a_n)n \in \m{C}^{\m{N}}$, par densité de $\m{Q}\left[i\right]$ dans $\m{C}$ :
 \[\ \forall n\in \m{N}\quad   \exists q_n \in \mathcal{B}(a_n,\frac{1}{n!})\cap \m{Q}\left[i\right] \] 
 Si on note $\varepsilon_n = q_n-a_n$ alors $\forall n\in \m{N} \quad   q_n = a_n + \varepsilon_n$ et $ |\varepsilon_n| \leq \frac{1}{n!} $,
  donc la série entiere $\sum_{n\geq 0}\varepsilon_nz^n$ a un rayon de convergence infini, de plus : \[\ \forall z \in \m{C} \quad \sum_{n\geq 0} q_nz^n =\sum_{n\geq 0} a_nz^n +\sum_{n\geq 0} \varepsilon_nz^n\] On a donc bien le résultat voulu puisque le terme le plus à droite converge toujours.\; ${\square}$\\
Pour $u = (u_n)_n$ une suite complexe, on note $f_u:z\mapsto \sum_{n = 0}^{\infty}u_nz^n$.
Notre hypothèse implique d'après ce lemme que \[\ \forall A \subset \mathcal{S}(0,R) \quad \exists q \in \m{Q}\left[i\right]^{\m{N}} \text{ tel que }  
f_q \text{ soit de rayon $R$ et } C(f_q) = A \]
 Donc l'application $ \varphi : \left\{\begin{matrix} \m{Q}\left[i\right]^{\m{N}} &\to & \mathcal{P(S(}0,R)) \\ q &\mapsto & C(f_q) \end{matrix} \right. \;\;$
  est surjective. Or on peut montrer avec une bijection de $\m{N}$ dans $\m{Q}$ et l'équipotence de $\m{Q}\left[i\right] $ 
  et $\m{Q}$ que $\m{Q}\left[i\right]^{\m{N}}$ est en bijection avec $\m{N}^{\m{N}}$. De plus $\m{R}$ est en bijection avec 
  $\m{N}^{\m{N}}$, exhibons une surjection, (c'est suffisant pour notre preuve) : à un réel $x$ dont une écriture en base 11 est une suite $(x_n)_n$
   à valeurs dans $\{0,\ldots ,9,A\} $ on associe, si $(x_n)_n$ possède deux $A$ consécutifs ou un nombre fini de $A$, la suite nulle et sinon la suite 
   $(a_n)_n$ definit par : $\forall n \in \m{N}\quad a_n $ est l'entier dont l'écriture décimale est $x_{i_n +1} \ldots x_{i_{n+1}-1}$ où $i_n$ désigne le rang du $n$-ième A dans $(x_n)_n $. Cette application est facilement surjective, par composition, on dispose donc d'une surjection de $\m{R}$ dans $\mathcal{P(S}(0,R))$. Or $\psi :\left\{\begin{matrix} \left[0,1\right[ & \to & \mathcal{S}(0,R) \\ t &\mapsto & Re^{2i\pi t}\end{matrix}\right. $ est bijective et donc démontre, connaissant l'équipotence de $\left[0,1\right[$ et $\m{R}$, que $S(0,R)$ est en bijection avec $\m{R}$, il vient ensuite que $\mathcal{P(S}(0,R))$ est en bijection avec $\mathcal{P}(\m{R})$. On a donc grace à tout ce qui précède, une surjection de $\m{R}$ dans $\mathcal{P}(\m{R})$ ce qui est absurde d'après le théorème de Cantor.

}
%------------------------------------------------------------------------------------------------------------------------------------------------
%------------------------------------------------------------------------------------------------------------------------------------------------

\exercice{15 (D'après \cite{ulm2019})}{ $\\$

Calculer  \[\ \lim_{x\to\infty}\Biggl( \sum_{n=1}^{\infty} \Bigl(\frac{x}{n}\Bigr)^n\Biggr)^{\frac{1}{x}} \]
}{
Le $n^{-n}$ peut nous rappeler la fameuse et magnifique identité : 
\[\ \int_0^1 t^{-t}\,\mathrm{d}t = \sum_{n=1}^\infty n^{-n} \]
Baptisée ``Sophomore's dream'' ou ``rêve du deuxième année'' ( oui oui, elle est vraie !!)

On va enfait montrer qu'on a même : \[\ \forall x \in \mathbb{R} \quad \int_0^1 t^{-xt}\,\mathrm{d}t = \sum_{n=1}^\infty x^{n-1}n^{-n} \]
Pour $x=0 $ l'intégrale converge, pour $x\neq0$  $t^{-xt} = e^{-xt\ln(t)} \to 1 $ car $t\ln(t) \to 0$ quand $t \to 0$, l'intégrale converge également. Pour la série entière, son rayon de convergence est clairement infini.
Soit  $x \in \mathbb{R}$:
\begin{align*} \int_0^1 t^{-xt}\,\mathrm{d}t  &= \int_0^1 e^{-xt\ln(t)} \mathrm{d}t \\ & = \int_0^1\sum_{n=0}^{\infty}\frac{(-xt\ln(t))^n}{n!} \mathrm{d}t  \\  ^{(*)} &= \sum_{n=0}^{\infty}\int_0^1\frac{(-xt\ln(t))^n}{n!} \mathrm{d}t  \\ & = \sum_{n=0}^{\infty}\int_{+\infty}^0\frac{(-1)^nx^ne^{\frac{-un}{n+1}}(-u)^n}{n!(n+1)^n} \Bigl(\frac{-e^{-\frac{u}{n+1}}}{n+1}\Bigr)\mathrm{d}u \quad \text{ avec } t=e^{-\frac{u}{n+1}} 
\\ & = 
\sum_{n=0}^{\infty}\frac{x^n}{n!(n+1)^{n+1}}\int_0^{+\infty} e^{-u}u^n\mathrm{d}u 
\\ & = 
\sum_{n=0}^{\infty}\frac{x^n}{n!(n+1)^{n+1}}\Gamma(n+1)
\\ & = \sum_{n=0}^{\infty}\frac{x^n}{(n+1)^{n+1}}
\\ & = \sum_{n=1}^{\infty}x^{n-1}n^{-n}
\end{align*}
(*) Notons $f_n(t) = \frac{(-xt\ln(t))^n}{n!}$, $\forall n \in \mathbb{N} \quad f_n$ est cpm et intégrable sur [0,1], de plus $\sum f_n$ converge simplement vers $F_x:t\mapsto t^{-xt}$ qui est cpm. Enfin comme $t \in [0,1] \;\; f_n$ est de signe constant, notre calcul à partir de (*) justife donc la convergence de $\sum\int_0^1|f_n|$. L'échange série - intégrale est donc justifié.
Alors \[\ S(x) := \Biggl( \sum_{n=1}^{\infty} \Bigl(\frac{x}{n}\Bigr)^n\Biggr)^{\frac{1}{x}} = \;\; x^{\frac{1}{x}}\Biggl(\int_0^1 t^{-xt} \mathrm{d}t\Biggr)^{\frac{1}{x}} \] D'abord, $x^{\frac{1}{x}} = e^{\frac{\ln(x)}{x}} \to 1$ Puis pour $x \geq 1$ : \[\ \Biggl(\int_0^1 t^{-xt} \mathrm{d}t\Biggr)^{\frac{1}{x}}  = \Biggl(\int_0^1 (t^{-t})^x \mathrm{d}t\Biggr)^{\frac{1}{x}} = \|f\|_{_x} \] Où $f: t \mapsto t^{-t}$. On se rappelle alors de l'exercice affirmant que \[\ \|f\|_{_x} \xrightarrow[x \to \infty]{} \|f\|_{_\infty} = \sup_{t\in [0,1]}|f(t)| \] Avec une rapide étude de $f$ on trouve que sa borne sup vaut $e^{\frac{1}{e}}$. Ainsi, \[\ \lim_{x \to \infty}S(x) = e^{\frac{1}{e}}\quad _{_\square} \] 
}

\exercice{16 (D'après \cite{igor})}{

Soit (E,d) un espace métrique
 compact et \begin{math} 
f: E\rightarrow  E  
\end{math} continue  
telle que : \\ 
\[\
\forall x,y \in E\;\;\; d(f(x),f(y)) \geq d(x,y) \;\;\;\;\;\;(*)
\]\
Démontrer que f est bijective puis que f est une isométrie.
}{
Soient\(\ x,y \in E \)\ tels que \(\ 
f(x) = f(y) \)\ alors $ 0 = d(f(x),f(y)) \geq d(x,y) \geq 0 $
donc $x= y $, ainsi $f$ est injective.\\
Pour la surjectivité fixons $y\in E$ et remarquons que pour $x\in E$ et $n\in \mathbb{N}$ \[\
d(f^{n+1}(x),f^{n}(y)) \geq d(f(x),y)
\]\ On cherche donc $x$ tel que $d(f^{n+1}(x),f^{n}(y)) \rightarrow 0$.
Comme $E$ est compact on peut trouver  une extractrice $\phi$ et $l\in E$ telle que : \[\ f^{\phi(n)}(y)\rightarrow l
\]\  Posons $x_n = f^{\phi(n+1)-\phi(n)-1}(y)$ et $a_n = f^{\phi(n)+1}(x_n)$
 Alors 
\[\ a_n = 
f^{\phi(n)+1}(f^{\phi(n+1)-\phi(n)-1}(y)) = f^{\phi(n+1)}(y) \rightarrow l
\]\ 
Extrayons une deuxième fois : il existe une extractrice $\psi$ et $x \in E$ tels que : 
\[\ x_{\psi(n)} \rightarrow x
\]\ Comme 
\[\ f^{\phi(\psi(n))+1}(x_{\psi(n)}) \rightarrow l,  \]\ (c'est une suite extraite de $a_n$), on aimerait en conclure que : 
\[\ f^{\phi(\psi(n))+1}(x) \rightarrow l
\]\ Attention ici à ne pas essayer d'utiliser la continuité de $f^{\phi(\psi(n))+1}$ , la dépendance en $n$ nous en empèche.
Nous allons pour cela montrer le résultat suivant : 
\paragraph{Lemme}
Soient $u_n \in \mathbb{N}^{\mathbb{N}}$ tendant vers $+\infty$, $b_n \in E^{\mathbb{N}} $ tendant vers $b \in E$ et $l,l'$  dans $E$ tels que : 
\[\ f^{u_n}(b_n) \rightarrow l \;\;\;et\;\;\;  f^{u_n}(b) \rightarrow l' \] Alors $l = l'$
\paragraph{Preuve}
Soit $\epsilon > 0$ : \[\ 
   \forall k,n\in \mathbb{N} \;\;\; d(f^{u_n}(b),l) \leq d(f^{u_n}(b),l') + d(l',f^{u_k}(b))  + d(f^{u_k}(b),f^{u_k}(b_n))+d(f^{u_k}(b_n),l) \;\;\;\;(1)\]\ De plus, en appliquant (*), si $u_n \geq u_k$:  
\[\  d(f^{u_k}(b_n),l) \leq d(f^{u_n}(b_n),f^{u_n-u_k}(l)) \leq d(f^{u_n}(b_n),l) + d(l,f^{u_n-u_k}(l))  \leq d(f^{u_n}(b_n),l) + d(f^{u_k}(l),f^{u_n}(l)) \]\
Par compacité, on dispose d'une extractrice $\phi$ et de $l_1 \in E$ telle que $f^{u_{\phi(p)}}(l)  \rightarrow l_1 \\ $
Finalement, en injectant dans (1) : \[\ d(f^{u_n}(b),l) \leq d(f^{u_n}(b),l') + d(l',f^{u_k}(b))  + d(f^{u_k}(b),f^{u_k}(b_n))+ d(f^{u_n}(b_n),l) + d(f^{u_k}(l),l_1) + d(l_1,f^{u_n}(l)) \]\
On peut donc  trouver $N_1 \in \mathbb{N} $ tel que $\forall p \geq N_1 \;\;\;d(f^{u_{\phi(p)}}(l),l_1) \leq \epsilon $ Il existe aussi $N_2 \in \mathbb{N} $ tel que $\forall p \geq N_2 \;\;\;d(l',f^{u_p}(b)) \leq \epsilon$. Fixons donc $ p_1 \geq max(N1,N2) $ et posons $ k = \phi(p_1) $ Ainsi $k \geq p_1 \geq N_2 $ et $k\geq N_1$ Donc \[\ \forall n \in \mathbb{N}\;\;\; d(f^{u_n}(b),l) \leq 2\epsilon + d(f^{u_n}(b),l')   + d(f^{u_k}(b),f^{u_k}(b_n))+ d(f^{u_n}(b_n),l) +  d(l_1,f^{u_n}(l)) \]\ Maintenant que k est fixé, on peut utiliser la continuité de $f^{u_k} \; : \;\\ \exists N_3 \in \mathbb{N} \;\;\; \forall p \geq N_3\;\;\; d(f^{u_k}(b),f^{u_k}(b_n)) \leq \epsilon \;\;\;\;\;\;  $ ($N_3$ dépend de k)$
\\ \exists N_4 \in \mathbb{N} \;\;\; \forall p \geq N_4\;\;\; d(f^{u_p}(b_p),l) \leq \epsilon \\ \\$ Soit donc $p_2 \geq max(N_1,N_2,N_3,N_4)$ et $n := \phi(p_2)  $ vérifiant $u_n \geq u_k \;\;\;\; $ (possible car $u$ tends vers $+\infty) \\ $ donc $ n \geq max(N_1,N_2,N_3,N_4) \\ $ On peut conclure : \[\  d(f^{u_n}(b),l) \leq 6\epsilon\]\ Ainsi $l$ est une valeur d'adhérence de $(f^{u_n}(b))_n$ et comme elle converge vers $l'$ on a bien $l=l'$ $_\Box \\ \\$

Pour utiliser notre lemme, on a donc besoin que $f^{\phi(\psi(n))+1}(x)$ converge,
 ce n'est pas forcément le cas, extrayons : il existe $ \gamma $ une extractrice
  et $l' \in E$ tels que \[\ f^{\phi(\psi(\gamma(n)))+1}(x) \rightarrow l' \]\ 
  Notons $u_n = \phi(\psi(\gamma(n)))$. On a : \[\ f^{u_n+1}(x_{\psi(\gamma(n))})
   \rightarrow l \;\;\;\; et \;\;\;\; x_{\psi(\gamma(n))} \rightarrow x\]\ Donc par
    le lemme : \[\ f^{u_n+1}(x) \rightarrow l \]\ On obtient donc
     \[\ d(f^{u_n + 1}(x),f^{u_n}(y)) \geq d(f(x),y) \] En passant 
     a la limite : \[\ d(f(x),y) \leq d(l,l) = 0 \] Ainsi $f(x) = y $, 
     ce qui démontre la surjectivité de $f$.  \[\ \]

Démontrons maintenant que $f$ est une isométrie : $\\ $ Comme $f$ est bijective on peut maintenant écrire : \[\ \forall x,y \in E \;\; d(x,y) \geq d(f^{-1}(x),f^{-1}(y))\;\;\;\;\;\;\;\; (**) \]
Pour $x,y \in E$ posons $\forall n \in \mathbb{N} \;\; d_n = d(f^{-n}(x),f^{-n}(y))$. D'après (**), $ (d_n)$ est décroissante, de plus elle est minorée par 0, elle est donc convergente, notons $l$ sa limite. Comme $E$ est compact il existe $\gamma , \psi $ des extractrices et $a,b\in E$ tels que \[\
f^{-\gamma(n)}(x) \rightarrow a, \;\;\; f^{-\gamma(\psi(n))}(y) \rightarrow b
\] Notons $\phi = \gamma \circ \psi$. Alors comme $d_{\phi(n)} \rightarrow l $ on a : $d(a,b) = l$, de plus par continuité de $f$ : \[\ f^{-\phi(n) + 1}(x) \rightarrow f(a), \;\;\; f^{-\phi(n) + 1}(y) \rightarrow f(b) \] Donc \[\ d(f^{-\phi(n) + 1}(x),f^{-\phi(n) + 1}(y)) \rightarrow d(f(a),f(b)) \]
Mais $d_{\phi(n) - 1}$ tends aussi vers $l$ donc  \[\ d(f(a),f(b)) = d(a,b) = l\] Ce qui ressemble pas mal a ce q'on cherche, on le veut pour tout $a,b \in E $. Fixons donc $a,b \in E$. Il suffirait donc d' avoir l' existence de $ x,y\in E$ ainsi que de $\gamma$ et $\psi$ des extractrices telles que \[\
f^{-\gamma(n)}(x) \rightarrow a, \;\;\; f^{-\gamma(\psi(n))}(y) \rightarrow b
\] Commençons avec $a$, ce n'est pas évident... On peut essayer d'exprimer $x$ en fonction de $a$ et naivement écrire ``$ x = f^{\gamma(n)}(a)"$ ce qui n'a pas de sens, mais $(f^{n}(a))_n$ est bien une suite de $E$ dont on peut donc extraire une suite convergente notons justement, pour voir $\gamma$ l'extractrice et  $x$ la limite :  \[\ f^{\gamma(n)}(a) \rightarrow x \] Notons $x_n = f^{\gamma(n)}(a)$. Alors comme $f^{-\gamma(n)}(x_n) = a$ on a : \[\ f^{-\gamma(n)}(x_n) \rightarrow a \;\;\; et \;\;\; x_n \rightarrow x\] Cela nous rappelle notre lemme, il nous manque l'hypothèse : ``$f^{-\gamma(n)}(x)$ converge '', mais quitte a extraire et a remplacer $\gamma$ par $\gamma \circ \gamma'$ par exemple, on peut la supposer vraie. Autre problème : la suite $(-\gamma(n))_n $ ne tends pas vers $+\infty \\ \\$
  Adaptons notre lemme aux suites de $\mathbb{Z}^{\mathbb{N}}$ qui tendent vers $-\infty$. En regardant la preuve on voit que l'hypothèse ``$u_n \in \mathbb{N}^{\mathbb{N}}$ '' n'est utilisé que pour avoir la continuité de $f^{u_k}$ car $f^{-1}$ n'est a priori pas continue, sauf qu'en fait si car elle est directement 1-lipschitzienne par (**) on peut donc prendre $u \in \mathbb{Z}^{\mathbb{N}}$. L'hypothèse ``$u_n \rightarrow +\infty$'' par contre  est nécessaire pour trouver $n$ vérifiant $u_n \geq u_k$ et appliquer la majoration $d(f^{u_k}(b_n),l) \leq d(f^{u_n}(b_n),f^{u_n-u_k}(l))$. Bon... reprenons les mêmes notations et changeons la première ligne en : \[\ 
   \forall n\in \mathbb{N} \;\;\; d(f^{u_n}(b),l) \leq d(f^{u_n}(b),f^{u_n}(b_n))+d(f^{u_n}(b_n),l) \;\;\;\;\]\ Dès que $u_n$ est négative, $d(f^{u_n}(b),f^{u_n}(b_n)) \leq d(b,b_n)$ en itérant (**). Mais comme $b_n \rightarrow b$ et $f^{u_n}(b_n) \rightarrow l$ on a directement $f^{u_n}(b) \rightarrow l$ puis $l = l'$. Donc c'est bon, c'était plus simple comme ça !  $ \\ \\ $ On peut donc utiliser cette version du  lemme et conclure que $f^{-\gamma(n)}(x)
   \rightarrow a $.
   Tout se passe exactement pareil pour trouver $y$ et $\psi$ si  on commence par extraire de la suite $(f^{\gamma(n)}(b))_n$. Cela conclut la preuve.
}

\exercice{17 (Preuve topologique de Cayley-Hamilton)}{

Soit $A\in \mn{C}$,\\
1) Démontrer que si $A$ est diagonalisable, alors $\chi_{_A}(A) =0$.\\
2) En déduire que $\chi_{_A}(A)=0$.
}
{
1) On note $A= PDP^{-1}$ où $D$ est diagonale, P inversible.
On a facimement $\chi_{_A}(A) = \chi_{_D}(A) = \chi_{_D}(D)$. 
Mais cette matrice est diagonale, et s'écrit $\prod_{i=1}^n(D-\lambda_i I_n)$. Où $Sp(A) = \{ \lambda _1,\ldots ,\lambda _n\}$. Et comme les coefficients diagonaux de $D$ sont aussi les valeurs propres de $A$, on peut voir que ce produit de matrices diagonales est nul.\\
2) On va conclure par densité des matrices diagonalisables dans $\mn{C}$ : on dispose d'une suite $(A_n)_n$ de matrices diagonalisables qui tends vers $A$. Cela ne dépend pas des normes, en dimension finie. Il suffit d'après 1) de montrer que $\chi_{_{A_n}}(A_n) \to \car{A}(A)$ car il s'agirait alors de la limite de la suite nulle.
L'application $f:M \mapsto \car{M}$ est continue puisque $ \car{M} = \mathrm{det}(XI_n-M)$ et le déterminant est polynomial en les coefficients de la matrice. Choisissons des normes qui nous arrangent : la norme infinie sur $\mn{C}$, on note $B = \mathcal{B}_f(A,1)$ et on pose pour $P\in \m{C}_n\left[X\right] \quad \|P\| := \mathrm{sup}_{M\in B}\|P(M)\|_{\infty}$. Le sup est bien définit car $P$ est continue sur le compact $B$ donc majorée. Montrons qu'on définit ainsi une norme : \\ 
Soient $P,Q \in \m{C}_n\left[X\right],\; \lambda \in \m{C}, $\\
$\bullet $ Homgénéité : $\|\lambda P\| = \mathrm{sup}_{M\in B}\|\lambda P(M)\|_{\infty}= \mathrm{sup}_{M\in B}|\lambda| \| P(M)\|_{\infty}= |\lambda|\n{P}$.\\
$\bullet$ Inégalité triangulaire :
$\forall M\in B \quad \n{P(M)+Q(M)}_{\infty}\leq \n{P(M)}_{\infty}+\n{Q(M)}_{\infty} \leq \n{P} + \n{Q} $ par passage au sup : $\n{P+Q}\leq \n{P} +\n{Q} $.\\
$\bullet$ Definition : Supposons $\n{P} =0$, $P$ est donc nul sur $B$. Par densité de $\gln{C}$ dans $\mn{C}$, on peut trouver $U \in \gln{C} \cap \mathring{B} $. Soit $\lambda$ une valeur propre de $U$ (il en existe toujours car son polynome caractéristique est scindé sur $\m{C}$) et $X$ un vecteur propre associé, comme $U$ est inversible, nécéssairement $\lambda \neq 0$. Puisque $\mathring{B}$ est ouvert, il existe $\varepsilon >0 $ tel que $\mathcal{B}(U,\varepsilon)\subset \mathring{B}$ En particulier $\forall t\in \left[0,\varepsilon \right[ \quad P(tU) = 0$. Puis pour $t\in \left[0,\varepsilon \right[ $ : \begin{align*}
tUX &= t\lambda X\\
P(tU)X &= P(t\lambda)X = 0
\end{align*}
Or $X \neq 0$ donc $P(t\lambda) =0$, $\lambda$ étant non nul, $P$ admet une infinité de racines dans $\m{C}$, donc $P=0$. (Remarquons que cela définit même une norme sur $\m{C}\left[X\right] $.)\\
\\
Puisque $f$ est continue on a $\car{A_n} \to \car{A}$ pour $\n{.}$ i.e $\n{\car{A_n}-\car{A}} \to 0$ i.e la suite de fonctions $(\car{A_n})_n$ converge uniformément vers $\car{A}$ sur $B$. Donc pour $n$ suffisament grand, $A_n\in B $ et \begin{align*}\n{\car{A_n}(A_n)-\car{A}(A)}_{\infty} &\leq \n{\car{A_n}(A_n)-\car{A}(A_n)}_{\infty} + \n{\car{A}(A_n)-\car{A}(A)}_{\infty}\\ &\leq \n{\car{A_n}-\car{A}} + \n{\car{A}(A_n)-\car{A}(A)}_{\infty} \end{align*}
On a vu que le terme de gauche tend vers 0, celui de droite tend aussi vers 0 par continuité de $M\mapsto \car{A}(M)$ comme polynome.
On a bien $\car{A_n}(A_n) \to \car{A}(A)$. 
}


\printbibliography %Prints bibliography

\end{document} 