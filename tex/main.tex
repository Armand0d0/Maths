
\documentclass[letterpaper,10pt]{article}

% Language setting
% Replace `english' with e.g. `spanish' to change the document language
\usepackage[utf8]{inputenc}
\usepackage[T1]{fontenc}
\usepackage[english]{babel}

% Set page size and margins
% Replace `letterpaper' with `a4paper' for UK/EU standard size
\usepackage[letterpaper,top=2cm,bottom=2cm,left=3cm,right=3cm,marginparwidth=1.75cm]{geometry}

% Useful packages
\usepackage{amsmath}
\usepackage{amssymb}
\usepackage{stmaryrd}
%\usepackage{graphicx}

\usepackage{biblatex} %Imports biblatex package
\usepackage[colorlinks=true, allcolors=blue]{hyperref}

\addbibresource{bibliography.bib} %Import the bibliography file


\title{Exos sympas}
\author{Armand Perrin}

\begin{document}

\maketitle%---------------------------------------------------------------

\newcommand{\mn}[1]{\ensuremath{\mathcal{M}_n(\mathbb{#1})}}

\paragraph{Exercice 1*:}
Soit $(G,.)$ un groupe fini dont tous les élément sont d'ordre 2, montrer que son
cardinal est une puissance de 2.

\paragraph{Exercice 2*(d'après \cite{igor}):}
Soit $\mathbb{K}$ un corps et $A_1, \ldots,A_p$ un ensemble de $p$ matrices de $GL_n(\mathbb{K})$ stable
par produit, montrer que \[\ tr\Biggl(\sum_{i=1}^p A_i\Biggr) \equiv 0\left[p\right] \]

\paragraph{Exercice 3*:}
Soient $a,b\in \mathbb{R},\;\; a<b$ et $f:\left[a,b\right] \to \mathbb{R}$ continue 
sur $\left[a,b\right]$, dérivable sur $\left]a,b\right[$ nulle en $a$ et $b$.
 Soit $\lambda \in \mathbb{R}$.
Démontrer que $f' + \lambda f$ s'annule sur $\left]a,b\right[$. 


\paragraph{Exercice 4:}
    Retrouver le binome de Newton à l'aide de la formule de Leibniz.

\paragraph{Exercice 5 (Bézout dans \mn{Z}):}
Soient $A,B \in \mn{Z}$ telles que $ det(A) \wedge det(B) = 1$, montrer qu'il
existe $U,V \in \mn{Z}$ telles que : \[\ AU+BV = I_n\]

\paragraph{Exercice 5*:}
Soit $M \in \mn{C}$, démontrer que $M$ est diagonalisable si et seulement si
sa classe de simlitude est fermée.

\paragraph{Exercice 6**(Théorème de Maschke):}
Soit $E$ un espace vectoriel de dimension finie et G un sous groupe fini de 
$GL(E)$. Démontrer que tout sous espace vectoriel de $E$ stable par tous les éléments de $G$ admet un 
supplémentaire également stable par tout les élément de $G$.

\paragraph{Exercice 7**:} Démontrer que $GL_n(\mathbb{C})$ est connexe par
 arcs.

\paragraph{Exercice 8**(ENS Lyon 2014):}

Determiner une CNS sur $A$ et $B$ dans $\mn{C})$ pour que \[\
\forall k \in \mathbb{N} \quad tr(A^k) = tr(B^k) \quad (*) \]
\paragraph{Solution :}
Pour commencer simplifions le problème et cherchons les matrices $A$ qui vérifient \[\
\forall k \in \mathbb{N^*}\quad tr(A^k) = 0
\quad (*) \] C'est classique, on commence par observer que toutes les matrices 
nilpotentes fonctionnent, puis on montre que ce sont les seules :
 démontrons que si $A \in \mn{C}) $ verifie $(*)$ alors son
  spectre est réduit à \{0\}. Notons $ \lambda_1,..., \lambda_r$ les valeurs propres non nulles de $A$ où les $\lambda_i$ sont distinct 2 à 2, $n_i \geq 1$ la multiplicité de $\lambda_i$ et $n_0 \geq 0$ la multiplicité de 0. On raisonne par l'absurde en supposant $r \geq 1$,
$(**)$ pour $k\in \llbracket 0, r \rrbracket $ donne : 
\begin{alignat*}{6}
n_0 + &n_1\lambda_1^0 &+\ldots &+n_r\lambda_r^0 & = n\\
&n_1\lambda_1 &+\ldots&+n_r\lambda_r  &= 0\\
&n_1\lambda_1^2 &+\ldots&+n_r\lambda_r^2 &= 0 \\
&\vdots &\vdots  & & \vdots \\
&n_1\lambda_1^{r} &+\ldots&+n_r\lambda_r^{r} &= 0
\end{alignat*}
Donc 
\[\ \begin{pmatrix}
1 & 1 &\ldots & 1 \\
0^1 & \lambda_1^1 & \ldots & \lambda_r^1 \\
\vdots & \vdots & \vdots & \vdots \\
0^r & \lambda_1^r & \ldots & \lambda_r^r \\
\end{pmatrix}\begin{pmatrix}n_0-n \\ n_1\\ \vdots \\ n_r
\end{pmatrix}
=\begin{pmatrix}0 \\ 0\\ \vdots \\ 0
\end{pmatrix}\]
Il s'agit d'un système type Vandermonde  inversible,
on en déduit $\forall i \in\llbracket 1, r\rrbracket \;\; n_i = 0 $ ce qui est absurde, donc $r = 0$ et $Sp(A) =$ \{0\}

Pour le cas général ont peut remarquer que si $A$ et $B$ ont le même polynome
 caractéristique alors elles vérifient $(*)$ de plus la réciproque est vraie 
 si $B=0$ d'après ce qui précède. On va montrer qu'elle est toujours vraie. 
 Pour une matrice $M$ notons $\chi_{_M}$ son polynome caractéristique et $\Pi{_M}$ 
 son polynome minimal. Fixons $A$ et $B$ dans $\mn{C})$ vérifiant $(*) $.
  D'abord, la linéarité de la trace permet d'écrire : \[\ \forall P \in 
  \mathbb{C}\left[X\right] \quad tr(P(A)) = tr(P(B)) \] 
Soit $k \in \mathbb{N}^*$ pour  $P = \chi_{_B}^k$ on a donc 
d'après Cayley-Hamilton : \[\ tr(\chi_{_B}^k(A)) = 0 \] La matrice
 $\chi_{_B}(A)$ est donc nilpotente, donc $\chi_{_B}^n$ annule $A$, donc
  $\Pi_{A} | \chi_{_B}^n$ et donc $Sp(A) \subset Sp(B)$. Par symétrie on 
  obtient ensuite l'inclusion réciproque, donc $Sp(A) = Sp(B)$.

Il ne reste plus qu'a montrer que les multiplicités des valeurs propres sont les
 mêmes pour avoir $\chi_{_A} =\chi_{_B}$. Notons $\{\lambda_1,\ldots,\lambda_r\}$ 
 le spectre commun et $a_i, b_i$ la multiplicité de $\lambda_i$ dans $\chi_{_A}$ 
 et $\chi_{_B}$ respectivement. $(*)$ donne alors :
\begin{alignat*}{8}
 &a_1\lambda_1^0 &+ \ldots  +& a_r\lambda_r^0 &=&\; b_1\lambda_1^0 &+ \ldots  +& b_r\lambda_r^0 \\
&\vdots & & & \vdots & & \vdots &\\
 &a_1\lambda_1^{r-1} &+ \ldots  +& a_r\lambda_r^{r-1} &=&\; b_1\lambda_1^{r-1} &+ \ldots  +& b_r\lambda_r^{r-1} \\
\end{alignat*} Ce qui s'écrit :
\[\ \begin{pmatrix}
1  &\ldots & 1 \\
\lambda_1^1 & \ldots & \lambda_r^1 \\
\vdots & \vdots & \vdots \\
\lambda_1^{r-1} & \ldots & \lambda_r^{r-1} \\
\end{pmatrix}\begin{pmatrix}a_1-b_1 \\ a_2-b_2\\ \vdots \\ a_r-b_r
\end{pmatrix}
=\begin{pmatrix}0 \\ 0\\ \vdots \\ 0
\end{pmatrix}\] On retombe sur une matrice de Vandermonde inversible et on conclut que \[\ \forall i \in \llbracket 1,r \rrbracket \quad a_i= b_i \] ce qui conclut.


\paragraph{Exercice 9***(D'après \cite{ulm2019}):}    $\\$

Calculer  \[\ \lim_{x\to\infty}\Biggl( \sum_{n=1}^{\infty} \Bigl(\frac{x}{n}\Bigr)^n\Biggr)^{\frac{1}{x}} \]

\paragraph{Solution :}
Le $n^{-n}$ peut nous rappeler la fameuse et magnifique identité : 
\[\ \int_0^1 t^{-t}\,\mathrm{d}t = \sum_{n=1}^\infty n^{-n} \]
Baptisée ``Sophomore's dream'' ou ``rêve du deuxième année'' ( oui oui, elle est vraie !!)

On va enfait montrer qu'on a même : \[\ \forall x \in \mathbb{R} \quad \int_0^1 t^{-xt}\,\mathrm{d}t = \sum_{n=1}^\infty x^{n-1}n^{-n} \]
Pour $x=0 $ l'intégrale converge, pour $x\neq0$  $t^{-xt} = e^{-xt\ln(t)} \to 1 $ car $t\ln(t) \to 0$ quand $t \to 0$, l'intégrale converge également. Pour la série entière, son rayon de convergence est clairement infini.
Soit  $x \in \mathbb{R}$:
\begin{align*} \int_0^1 t^{-xt}\,\mathrm{d}t  &= \int_0^1 e^{-xt\ln(t)} \mathrm{d}t \\ & = \int_0^1\sum_{n=0}^{\infty}\frac{(-xt\ln(t))^n}{n!} \mathrm{d}t  \\  ^{(*)} &= \sum_{n=0}^{\infty}\int_0^1\frac{(-xt\ln(t))^n}{n!} \mathrm{d}t  \\ & = \sum_{n=0}^{\infty}\int_{+\infty}^0\frac{(-1)^nx^ne^{\frac{-un}{n+1}}(-u)^n}{n!(n+1)^n} \Bigl(\frac{-e^{-\frac{u}{n+1}}}{n+1}\Bigr)\mathrm{d}u \quad \text{ avec } t=e^{-\frac{u}{n+1}} 
\\ & = 
\sum_{n=0}^{\infty}\frac{x^n}{n!(n+1)^{n+1}}\int_0^{+\infty} e^{-u}u^n\mathrm{d}u 
\\ & = 
\sum_{n=0}^{\infty}\frac{x^n}{n!(n+1)^{n+1}}\Gamma(n+1)
\\ & = \sum_{n=0}^{\infty}\frac{x^n}{(n+1)^{n+1}}
\\ & = \sum_{n=1}^{\infty}x^{n-1}n^{-n}
\end{align*}
(*) Notons $f_n(t) = \frac{(-xt\ln(t))^n}{n!}$, $\forall n \in \mathbb{N} \quad f_n$ est cpm et intégrable sur [0,1], de plus $\sum f_n$ converge simplement vers $F_x:t\mapsto t^{-xt}$ qui est cpm. Enfin comme $t \in [0,1] \;\; f_n$ est de signe constant, notre calcul à partir de (*) justife donc la convergence de $\sum\int_0^1|f_n|$. L'échange série - intégrale est donc justifié.
Alors \[\ S(x) := \Biggl( \sum_{n=1}^{\infty} \Bigl(\frac{x}{n}\Bigr)^n\Biggr)^{\frac{1}{x}} = \;\; x^{\frac{1}{x}}\Biggl(\int_0^1 t^{-xt} \mathrm{d}t\Biggr)^{\frac{1}{x}} \] D'abord, $x^{\frac{1}{x}} = e^{\frac{\ln(x)}{x}} \to 1$ Puis pour $x \geq 1$ : \[\ \Biggl(\int_0^1 t^{-xt} \mathrm{d}t\Biggr)^{\frac{1}{x}}  = \Biggl(\int_0^1 (t^{-t})^x \mathrm{d}t\Biggr)^{\frac{1}{x}} = \|f\|_{_x} \] Où $f: t \mapsto t^{-t}$. On se rappelle alors de l'exercice affirmant que \[\ \|f\|_{_x} \xrightarrow[x \to \infty]{} \|f\|_{_\infty} = \sup_{t\in [0,1]}|f(t)| \] Avec une rapide étude de $f$ on trouve que sa borne sup vaut $e^{\frac{1}{e}}$. Ainsi, \[\ \lim_{x \to \infty}S(x) = e^{\frac{1}{e}}\quad _{_\square} \] 


\paragraph{Exercice 10***(D'après \cite{igor}):}

Soit (E,d) un espace métrique
 compact et \begin{math} 
f: E\rightarrow  E  
\end{math} continue  
telle que : \\ 
\[\
\forall x,y \in E\;\;\; d(f(x),f(y)) \geq d(x,y) \;\;\;\;\;\;(*)
\]\
Démontrer que f est bijective puis que f est une isométrie.
\paragraph{Solution :} 
Soient\(\ x,y \in E \)\ tels que \(\ 
f(x) = f(y) \)\ alors $ 0 = d(f(x),f(y)) \geq d(x,y) \geq 0 $
donc $x= y $, ainsi $f$ est injective.\\
Pour la surjectivité fixons $y\in E$ et remarquons que pour $x\in E$ et $n\in \mathbb{N}$ \[\
d(f^{n+1}(x),f^{n}(y)) \geq d(f(x),y)
\]\ On cherche donc $x$ tel que $d(f^{n+1}(x),f^{n}(y)) \rightarrow 0$.
Comme $E$ est compact on peut trouver  une extractrice $\phi$ et $l\in E$ telle que : \[\ f^{\phi(n)}(y)\rightarrow l
\]\  Posons $x_n = f^{\phi(n+1)-\phi(n)-1}(y)$ et $a_n = f^{\phi(n)+1}(x_n)$
 Alors 
\[\ a_n = 
f^{\phi(n)+1}(f^{\phi(n+1)-\phi(n)-1}(y)) = f^{\phi(n+1)}(y) \rightarrow l
\]\ 
Extrayons une deuxième fois : il existe une extractrice $\psi$ et $x \in E$ tels que : 
\[\ x_{\psi(n)} \rightarrow x
\]\ Comme 
\[\ f^{\phi(\psi(n))+1}(x_{\psi(n)}) \rightarrow l,  \]\ (c'est une suite extraite de $a_n$), on aimerait en conclure que : 
\[\ f^{\phi(\psi(n))+1}(x) \rightarrow l
\]\ Attention ici à ne pas essayer d'utiliser la continuité de $f^{\phi(\psi(n))+1}$ , la dépendance en $n$ nous en empèche.
Nous allons pour cela montrer le résultat suivant : 
\paragraph{Lemme}
Soient $u_n \in \mathbb{N}^{\mathbb{N}}$ tendant vers $+\infty$, $b_n \in E^{\mathbb{N}} $ tendant vers $b \in E$ et $l,l'$  dans $E$ tels que : 
\[\ f^{u_n}(b_n) \rightarrow l \;\;\;et\;\;\;  f^{u_n}(b) \rightarrow l' \] Alors $l = l'$
\paragraph{Preuve}
Soit $\epsilon > 0$ : \[\ 
   \forall k,n\in \mathbb{N} \;\;\; d(f^{u_n}(b),l) \leq d(f^{u_n}(b),l') + d(l',f^{u_k}(b))  + d(f^{u_k}(b),f^{u_k}(b_n))+d(f^{u_k}(b_n),l) \;\;\;\;(1)\]\ De plus, en appliquant (*), si $u_n \geq u_k$:  
\[\  d(f^{u_k}(b_n),l) \leq d(f^{u_n}(b_n),f^{u_n-u_k}(l)) \leq d(f^{u_n}(b_n),l) + d(l,f^{u_n-u_k}(l))  \leq d(f^{u_n}(b_n),l) + d(f^{u_k}(l),f^{u_n}(l)) \]\
Par compacité, on dispose d'une extractrice $\phi$ et de $l_1 \in E$ telle que $f^{u_{\phi(p)}}(l)  \rightarrow l_1 \\ $
Finalement, en injectant dans (1) : \[\ d(f^{u_n}(b),l) \leq d(f^{u_n}(b),l') + d(l',f^{u_k}(b))  + d(f^{u_k}(b),f^{u_k}(b_n))+ d(f^{u_n}(b_n),l) + d(f^{u_k}(l),l_1) + d(l_1,f^{u_n}(l)) \]\
On peut donc  trouver $N_1 \in \mathbb{N} $ tel que $\forall p \geq N_1 \;\;\;d(f^{u_{\phi(p)}}(l),l_1) \leq \epsilon $ Il existe aussi $N_2 \in \mathbb{N} $ tel que $\forall p \geq N_2 \;\;\;d(l',f^{u_p}(b)) \leq \epsilon$. Fixons donc $ p_1 \geq max(N1,N2) $ et posons $ k = \phi(p_1) $ Ainsi $k \geq p_1 \geq N_2 $ et $k\geq N_1$ Donc \[\ \forall n \in \mathbb{N}\;\;\; d(f^{u_n}(b),l) \leq 2\epsilon + d(f^{u_n}(b),l')   + d(f^{u_k}(b),f^{u_k}(b_n))+ d(f^{u_n}(b_n),l) +  d(l_1,f^{u_n}(l)) \]\ Maintenant que k est fixé, on peut utiliser la continuité de $f^{u_k} \; : \;\\ \exists N_3 \in \mathbb{N} \;\;\; \forall p \geq N_3\;\;\; d(f^{u_k}(b),f^{u_k}(b_n)) \leq \epsilon \;\;\;\;\;\;  $ ($N_3$ dépend de k)$
\\ \exists N_4 \in \mathbb{N} \;\;\; \forall p \geq N_4\;\;\; d(f^{u_p}(b_p),l) \leq \epsilon \\ \\$ Soit donc $p_2 \geq max(N_1,N_2,N_3,N_4)$ et $n := \phi(p_2)  $ vérifiant $u_n \geq u_k \;\;\;\; $ (possible car $u$ tends vers $+\infty) \\ $ donc $ n \geq max(N_1,N_2,N_3,N_4) \\ $ On peut conclure : \[\  d(f^{u_n}(b),l) \leq 6\epsilon\]\ Ainsi $l$ est une valeur d'adhérence de $(f^{u_n}(b))_n$ et comme elle converge vers $l'$ on a bien $l=l'$ $_\Box \\ \\$

Pour utiliser notre lemme, on a donc besoin que $f^{\phi(\psi(n))+1}(x)$ converge,
 ce n'est pas forcément le cas, extrayons : il existe $ \gamma $ une extractrice
  et $l' \in E$ tels que \[\ f^{\phi(\psi(\gamma(n)))+1}(x) \rightarrow l' \]\ 
  Notons $u_n = \phi(\psi(\gamma(n)))$. On a : \[\ f^{u_n+1}(x_{\psi(\gamma(n))})
   \rightarrow l \;\;\;\; et \;\;\;\; x_{\psi(\gamma(n))} \rightarrow x\]\ Donc par
    le lemme : \[\ f^{u_n+1}(x) \rightarrow l \]\ On obtient donc
     \[\ d(f^{u_n + 1}(x),f^{u_n}(y)) \geq d(f(x),y) \] En passant 
     a la limite : \[\ d(f(x),y) \leq d(l,l) = 0 \] Ainsi $f(x) = y $, 
     ce qui démontre la surjectivité de $f$.  \[\ \]

Démontrons maintenant que $f$ est une isométrie : $\\ $ Comme $f$ est bijective on peut maintenant écrire : \[\ \forall x,y \in E \;\; d(x,y) \geq d(f^{-1}(x),f^{-1}(y))\;\;\;\;\;\;\;\; (**) \]
Pour $x,y \in E$ posons $\forall n \in \mathbb{N} \;\; d_n = d(f^{-n}(x),f^{-n}(y))$. D'après (**), $ (d_n)$ est décroissante, de plus elle est minorée par 0, elle est donc convergente, notons $l$ sa limite. Comme $E$ est compact il existe $\gamma , \psi $ des extractrices et $a,b\in E$ tels que \[\
f^{-\gamma(n)}(x) \rightarrow a, \;\;\; f^{-\gamma(\psi(n))}(y) \rightarrow b
\] Notons $\phi = \gamma \circ \psi$. Alors comme $d_{\phi(n)} \rightarrow l $ on a : $d(a,b) = l$, de plus par continuité de $f$ : \[\ f^{-\phi(n) + 1}(x) \rightarrow f(a), \;\;\; f^{-\phi(n) + 1}(y) \rightarrow f(b) \] Donc \[\ d(f^{-\phi(n) + 1}(x),f^{-\phi(n) + 1}(y)) \rightarrow d(f(a),f(b)) \]
Mais $d_{\phi(n) - 1}$ tends aussi vers $l$ donc  \[\ d(f(a),f(b)) = d(a,b) = l\] Ce qui ressemble pas mal a ce q'on cherche, on le veut pour tout $a,b \in E $. Fixons donc $a,b \in E$. Il suffirait donc d' avoir l' existence de $ x,y\in E$ ainsi que de $\gamma$ et $\psi$ des extractrices telles que \[\
f^{-\gamma(n)}(x) \rightarrow a, \;\;\; f^{-\gamma(\psi(n))}(y) \rightarrow b
\] Commençons avec $a$, ce n'est pas évident... On peut essayer d'exprimer $x$ en fonction de $a$ et naivement écrire ``$ x = f^{\gamma(n)}(a)"$ ce qui n'a pas de sens, mais $(f^{n}(a))_n$ est bien une suite de $E$ dont on peut donc extraire une suite convergente notons justement, pour voir $\gamma$ l'extractrice et  $x$ la limite :  \[\ f^{\gamma(n)}(a) \rightarrow x \] Notons $x_n = f^{\gamma(n)}(a)$. Alors comme $f^{-\gamma(n)}(x_n) = a$ on a : \[\ f^{-\gamma(n)}(x_n) \rightarrow a \;\;\; et \;\;\; x_n \rightarrow x\] Cela nous rappelle notre lemme, il nous manque l'hypothèse : ``$f^{-\gamma(n)}(x)$ converge '', mais quitte a extraire et a remplacer $\gamma$ par $\gamma \circ \gamma'$ par exemple, on peut la supposer vraie. Autre problème : la suite $(-\gamma(n))_n $ ne tends pas vers $+\infty \\ \\$
  Adaptons notre lemme aux suites de $\mathbb{Z}^{\mathbb{N}}$ qui tendent vers $-\infty$. En regardant la preuve on voit que l'hypothèse ``$u_n \in \mathbb{N}^{\mathbb{N}}$ '' n'est utilisé que pour avoir la continuité de $f^{u_k}$ car $f^{-1}$ n'est a priori pas continue, sauf qu'en fait si car elle est directement 1-lipschitzienne par (**) on peut donc prendre $u \in \mathbb{Z}^{\mathbb{N}}$. L'hypothèse ``$u_n \rightarrow +\infty$'' par contre  est nécessaire pour trouver $n$ vérifiant $u_n \geq u_k$ et appliquer la majoration $d(f^{u_k}(b_n),l) \leq d(f^{u_n}(b_n),f^{u_n-u_k}(l))$. Bon... reprenons les mêmes notations et changeons la première ligne en : \[\ 
   \forall n\in \mathbb{N} \;\;\; d(f^{u_n}(b),l) \leq d(f^{u_n}(b),f^{u_n}(b_n))+d(f^{u_n}(b_n),l) \;\;\;\;\]\ Dès que $u_n$ est négative, $d(f^{u_n}(b),f^{u_n}(b_n)) \leq d(b,b_n)$ en itérant (**). Mais comme $b_n \rightarrow b$ et $f^{u_n}(b_n) \rightarrow l$ on a directement $f^{u_n}(b) \rightarrow l$ puis $l = l'$. Donc c'est bon, c'était plus simple comme ça !  $ \\ \\ $ On peut donc utiliser cette version du  lemme et conclure que $f^{-\gamma(n)}(x)
   \rightarrow a $.
   Tout se passe exactement pareil pour trouver $y$ et $\psi$ si  on commence par extraire de la suite $(f^{\gamma(n)}(b))_n$. Cela conclut la preuve.



\printbibliography %Prints bibliography

\end{document} 