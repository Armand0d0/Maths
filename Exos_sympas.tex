\documentclass{article}

% Language setting
% Replace `english' with e.g. `spanish' to change the document language
\usepackage[english]{babel}

% Set page size and margins
% Replace `letterpaper' with `a4paper' for UK/EU standard size
\usepackage[letterpaper,top=2cm,bottom=2cm,left=3cm,right=3cm,marginparwidth=1.75cm]{geometry}

% Useful packages
\usepackage{amsmath}
\usepackage{amssymb}

\usepackage{graphicx}
\usepackage[colorlinks=true, allcolors=blue]{hyperref}

\title{Exos sympas}
\author{Armand Perrin}

\begin{document}
\maketitle

\paragraph{Exercice :}

Soit (E,d) un espace métrique
 compact et \begin{math} 
f: E\rightarrow  E 
\end{math} continue 
telle que : \\ 
\[\
\forall x,y \in E\;\;\; d(f(x),f(y)) \geq d(x,y) \;\;\;\;\;\;(*)
\]\
Démontrer que f est bijective puis que f est une isométrie.
\paragraph{Solution :} 
Soient\(\ x,y \in E \)\ tels que \(\ 
f(x) = f(y) \)\ alors $ 0 = d(f(x),f(y)) \geq d(x,y) \geq 0 $
donc $x= y $, ainsi $f$ est injective.\\
Pour la surjectivité fixons $y\in E$ et remarquons que pour $x\in E$ et $n\in \mathbb{N}$ \[\
d(f^{n+1}(x),f^{n}(y)) \geq d(f(x),y)
\]\ On cherche donc $x$ tel que $d(f^{n+1}(x),f^{n}(y)) \rightarrow 0$.
Comme $E$ est compact on peut trouver  une extractrice $\phi$ et $l\in E$ telle que : \[\ f^{\phi(n)}(y)\rightarrow l
\]\  Posons $x_n = f^{\phi(n+1)-\phi(n)-1}(y)$ et $a_n = f^{\phi(n)+1}(x_n)$
 Alors 
\[\ a_n = 
f^{\phi(n)+1}(f^{\phi(n+1)-\phi(n)-1}(y)) = f^{\phi(n+1)}(y) \rightarrow l
\]\
Extrayons une deuxième fois : il existe une extractrice $\psi$ et $x \in E$ tels que : 
\[\ x_{\psi(n)} \rightarrow x
\]\ Comme 
\[\ f^{\phi(\psi(n))+1}(x_{\psi(n)}) \rightarrow l,  \]\ (c'est une suite extraite de $a_n$), on aimerait en conclure que : 
\[\ f^{\phi(\psi(n))+1}(x) \rightarrow l
\]\ Attention ici à ne pas essayer d'utiliser la continuité de $f^{\phi(\psi(n))+1}$ , la dépendance en $n$ nous en empèche.
Nous allons pour cela montrer le résultat suivant : 
\paragraph{Lemme}
Soient $u_n \in \mathbb{N}^{\mathbb{N}}$ tendant vers $+\infty$, $b_n \in E^{\mathbb{N}} $ tendant vers $b \in E$ et $l,l'$  dans $E$ tels que : 
\[\ f^{u_n}(b_n) \rightarrow l \;\;\;et\;\;\;  f^{u_n}(b) \rightarrow l' \] Alors $l = l'$
\paragraph{Preuve}
Soit $\epsilon > 0$ : \[\ 
   \forall k,n\in \mathbb{N} \;\;\; d(f^{u_n}(b),l) \leq d(f^{u_n}(b),l') + d(l',f^{u_k}(b))  + d(f^{u_k}(b),f^{u_k}(b_n))+d(f^{u_k}(b_n),l) \;\;\;\;(1)\]\ De plus, en appliquant (*), si $u_n \geq u_k$:  
\[\  d(f^{u_k}(b_n),l) \leq d(f^{u_n}(b_n),f^{u_n-u_k}(l)) \leq d(f^{u_n}(b_n),l) + d(l,f^{u_n-u_k}(l))  \leq d(f^{u_n}(b_n),l) + d(f^{u_k}(l),f^{u_n}(l)) \]\
Par compacité, on dispose d'une extractrice $\phi$ et de $l_1 \in E$ telle que $f^{u_{\phi(p)}}(l)  \rightarrow l_1 \\ $
Finalement, en injectant dans (1) : \[\ d(f^{u_n}(b),l) \leq d(f^{u_n}(b),l') + d(l',f^{u_k}(b))  + d(f^{u_k}(b),f^{u_k}(b_n))+ d(f^{u_n}(b_n),l) + d(f^{u_k}(l),l_1) + d(l_1,f^{u_n}(l)) \]\
On peut donc  trouver $N_1 \in \mathbb{N} $ tel que $\forall p \geq N_1 \;\;\;d(f^{u_{\phi(p)}}(l),l_1) \leq \epsilon $ Il existe aussi $N_2 \in \mathbb{N} $ tel que $\forall p \geq N_2 \;\;\;d(l',f^{u_p}(b)) \leq \epsilon$. Fixons donc $ p_1 \geq max(N1,N2) $ et posons $ k = \phi(p_1) $ Ainsi $k \geq p_1 \geq N_2 $ et $k\geq N_1$ Donc \[\ \forall n \in \mathbb{N}\;\;\; d(f^{u_n}(b),l) \leq 2\epsilon + d(f^{u_n}(b),l')   + d(f^{u_k}(b),f^{u_k}(b_n))+ d(f^{u_n}(b_n),l) +  d(l_1,f^{u_n}(l)) \]\ Maintenant que k est fixé, on peut utiliser la continuité de $f^{u_k} \; : \;\\ \exists N_3 \in \mathbb{N} \;\;\; \forall p \geq N_3\;\;\; d(f^{u_k}(b),f^{u_k}(b_n)) \leq \epsilon \;\;\;\;\;\;  $ ($N_3$ dépend de k)$
\\ \exists N_4 \in \mathbb{N} \;\;\; \forall p \geq N_4\;\;\; d(f^{u_p}(b_p),l) \leq \epsilon \\ \\$ Soit donc $p_2 \geq max(N_1,N_2,N_3,N_4)$ et $n := \phi(p_2)  $ vérifiant $u_n \geq u_k \;\;\;\; $ (possible car $u$ tends vers $+\infty) \\ $ donc $ n \geq max(N_1,N_2,N_3,N_4) \\ $ On peut conclure : \[\  d(f^{u_n}(b),l) \leq 6\epsilon\]\ Ainsi $l$ est une valeur d'adhérence de $(f^{u_n}(b))_n$ et comme elle converge vers $l'$ on a bien $l=l'$ $_\Box \\ \\$

Pour utiliser notre lemme, on a donc besoin que $f^{\phi(\psi(n))+1}(x)$ converge, ce n'est pas forcément le cas, extrayons : il existe $ \gamma $ une extractrice et $l' \in E$ tels que \[\ f^{\phi(\psi(\gamma(n)))+1}(x) \rightarrow l' \]\ Notons $u_n = \phi(\psi(\gamma(n)))$. On a : \[\ f^{u_n+1}(x_{\psi(\gamma(n))}) \rightarrow l \;\;\;\; et \;\;\;\; x_{\psi(\gamma(n))} \rightarrow x\]\ Donc par le lemme : \[\ f^{u_n+1}(x) \rightarrow l \]\ On obtient donc \[\ d(f^{u_n + 1}(x),f^{u_n}(y)) \geq d(f(x),y) \] En passant a la limite : \[\ d(f(x),y) \leq d(l,l) = 0 \] Ainsi $f(x) = y $, ce qui démontre la surjectivité de f.  \[\ \]

Démontrons maintenant que $f$ est une isométrie : $\\ $ Comme $f$ est bijective on peut maintenant écrire : \[\ \forall x,y \in E \;\; d(x,y) \geq d(f^{-1}(x),f^{-1}(y))\;\;\;\;\;\;\;\; (**) \]
Pour $x,y \in E$ posons $\forall n \in \mathbb{N} \;\; d_n = d(f^{-n}(x),f^{-n}(y))$. D'après (**), $ (d_n)$ est décroissante, de plus elle est minorée par 0, elle est donc convergente, notons $l$ sa limite. Comme $E$ est compact il existe $\gamma , \psi $ des extractrices et $a,b\in E$ tels que \[\
f^{-\gamma(n)}(x) \rightarrow a, \;\;\; f^{-\gamma(\psi(n))}(y) \rightarrow b
\] Notons $\phi = \gamma \circ \psi$. Alors comme $d_{\phi(n)} \rightarrow l $ on a : $d(a,b) = l$, de plus par continuité de $f$ : \[\ f^{-\phi(n) + 1}(x) \rightarrow f(a), \;\;\; f^{-\phi(n) + 1}(y) \rightarrow f(b) \] Donc \[\ d(f^{-\phi(n) + 1}(x),f^{-\phi(n) + 1}(y)) \rightarrow d(f(a),f(b)) \]
Mais $d_{\phi(n) - 1}$ tends aussi vers $l$ donc  \[\ d(f(a),f(b)) = d(a,b) = l\] Ce qui ressemble pas mal a ce q'on cherche, on le veut pour tout $a,b \in E $. Fixons donc $a,b \in E$. Il suffirait donc d' avoir l' existence de $ x,y\in E$ ainsi que de $\gamma$ et $\psi$ des extractrices telles que \[\
f^{-\gamma(n)}(x) \rightarrow a, \;\;\; f^{-\gamma(\psi(n))}(y) \rightarrow b
\] Commençons avec $a$, ce n'est pas évident... On peut essayer d'exprimer $x$ en fonction de $a$ et naivement écrire ``$ x = f^{\gamma(n)}(a)"$ ce qui n'a pas de sens, mais $(f^{n}(a))_n$ est bien une suite de $E$ dont on peut donc extraire une suite convergente notons justement, pour voir $\gamma$ l'extractrice et  $x$ la limite :  \[\ f^{\gamma(n)}(a) \rightarrow x \] Notons $x_n = f^{\gamma(n)}(a)$. Alors comme $f^{-\gamma(n)}(x_n) = a$ on a : \[\ f^{-\gamma(n)}(x_n) \rightarrow a \;\;\; et \;\;\; x_n \rightarrow x\] Cela nous rappelle notre lemme, il nous manque l'hypothèse : ``$f^{-\gamma(n)}(x)$ converge '', mais quitte a extraire et a remplacer $\gamma$ par $\gamma \circ \gamma'$ par exemple, on peut la supposer vraie. Autre problème : la suite $(-\gamma(n))_n $ ne tends pas vers $+\infty \\ \\$  Adaptons notre lemme aux suites de $\mathbb{Z}^{\mathbb{N}}$ qui tendent vers $-\infty$. En regardant la preuve on voit que l'hypothèse ``$u_n \in \mathbb{N}^{\mathbb{N}}$ " n'est utilisé que pour avoir la continuité de $f^{u_k}$ car $f^{-1}$ n'est a priori pas continue, sauf qu'en fait si car elle est directement 1-lipschitzienne par (**) on peut donc prendre $u \in \mathbb{Z}^{\mathbb{N}}$. L'hypothèse ``$u_n \rightarrow +\infty$" par contre  est nécessaire pour trouver $n$ vérifiant $u_n \geq u_k$ et appliquer la majoration $d(f^{u_k}(b_n),l) \leq d(f^{u_n}(b_n),f^{u_n-u_k}(l))$. Bon... reprenons les mêmes notations et changeons la première ligne en : \[\ 
   \forall n\in \mathbb{N} \;\;\; d(f^{u_n}(b),l) \leq d(f^{u_n}(b),f^{u_n}(b_n))+d(f^{u_n}(b_n),l) \;\;\;\;\]\ Dès que $u_n$ est négative, $d(f^{u_n}(b),f^{u_n}(b_n)) \leq d(b,b_n)$ en itérant (**). Mais comme $b_n \rightarrow b$ et $f^{u_n}(b_n) \rightarrow l$ on a directement $f^{u_n}(b) \rightarrow l$ puis $l = l'$. Donc c'est bon, c'était plus simple comme ça !  $ \\ \\ $ On peut donc utiliser cette version du  lemme et conclure que $f^{-\gamma(n)}(x)
   \rightarrow a $.
   Tout se passe exactement pareil pour trouver $y$ et $\psi$ si  on commence par extraire de la suite $(f^{\gamma(n)}(b))_n$. Cela conclut la preuve.
%\bibliographystyle{alpha}
%\bibliography{sample}

\end{document}
